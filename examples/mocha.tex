\documentclass[12pt]{article}

% Imports the catppuccino mocha theme
% from the directory above. Actual implementation
% wouldn't need the import package unless the theme
% and the document are in different directories.
\usepackage{import}
\import{../}{catppuccin_mocha.sty}

% Removes padding above title
\usepackage{titling}
\setlength{\droptitle}{-10em}

% Margin package
\usepackage[margin=1.5in]{geometry}

% Font package
\usepackage[T1]{fontenc}
\usepackage{inconsolata}

% Code snippets package
\usepackage{listings}
\lstdefinestyle{mystyle}{
  language={Ruby},
  breaklines=true,
  showstringspaces=false,
  breakatwhitespace=true,
  stringstyle = {\color{mochaGreen}},
  commentstyle={\color{mochaOverlay1}},
  basicstyle = {\small\color{mochaText}\ttfamily},
  keywordstyle = {\color{mochaMauve}},
  keywordstyle = [2]{\color{mochaBlue}},
  keywordstyle = [3]{\color{mochaYellow}},
  keywordstyle = [4]{\color{mochaLavender}},
  keywordstyle = [5]{\color{mochaPeach}},
  keywordstyle = [6]{\color{mochaTeal}},
  otherkeywords = {<, ||, =, ?},
  morekeywords = [2]{new, create, present, email, description, creator, protect_from_forgery, before_action},
  morekeywords = [3]{PageController, ApplicationController, Page},
  morekeywords = [4]{@page},
  morekeywords = [5]{exception, do_some_for_pages, @page, @admin},
  morekeywords = [6]{<, ||, =, ?},
}

% Set the title and author, utilizing mochaPink
\title{ \Huge \textbf{\textcolor{mochaPink}{Catppuccin Mocha Theme for} \textcolor{mochaLavender}{\LaTeX{} Documents}} \vspace{-3em}}
\date{}

% Start our dock
\begin{document}

\maketitle

\section{\textcolor{mochaSky}{Simple Paragraph}}
\textcolor{mochaYellow}{This is what a simple paragraph looks like.} You can see that all the colors in this pallet complement each other and maintain an elegance all within the same proximity. The default text color is called \textbf{\textcolor{mochaGreen}{mochaText}} while the background color is called \textbf{\textcolor{mochaGreen}{mochaBase}}.

\section{\textcolor{mochaSky}{Math}}

You even can make math formulas, equations, and examples look nicer and more legible:


\[\large{
    \textcolor{mochaBlue}{\lim_{{n \to \infty}} \textcolor{mochaPeach}{\int_{\textcolor{mochaRed}{a}}^{\textcolor{mochaRed}{b}} \frac{\textcolor{mochaGreen}{\sin}(\textcolor{mochaYellow}{nx})}{\textcolor{mochaYellow}{x}} \, \textcolor{mochaTeal}{dx}}}
}\]

\tiny{(Or just color dump like I did)}

\section{\textcolor{mochaSky}{Code Snippets}}

\begin{lstlisting}[language=Ruby,style=mystyle, caption={A ruby on rails code sample}]
class PageController < ApplicationController
  protect_from_forgery with: :exception
  before_action :do_some_for_pages

  '''
  This #new action creates a new page object and sets the
  '''
  def new
    @page = Page.new
    creator = current_user || @admin

    # Check if a creator is present
    if creator.present?
      @page.creator = creator.email
    end

    @page.description = 'This is the default description'
  end
end
\end{lstlisting}

\end{document}


