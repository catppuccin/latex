\documentclass[12pt,a4paper]{article}

% Imports the catppuccin theme, using the latte flavor,
% from the directory above. Actual implementation
% wouldn't need the import package unless the theme
% and the document are in different directories.
\usepackage{import}
\usepackage[\jobname,styleAll]{catppuccinpalette}

\usepackage{hyperref}

% Removes padding above title
\usepackage{titling}
% \setlength{\droptitle}{-10em}

% Margin package
\usepackage[margin=1.5in]{geometry}

% Font package
\usepackage[T1]{fontenc}
\usepackage{inconsolata}

% Code snippets package
\usepackage{listings}
% Add code highlighting specific for my Ruby on Rails snippet
% You'd want to follow something similar below to add your own keyword highlighting
\lstdefinestyle{ruby_on_rails}{
	language={Ruby},
	breaklines=true,
	showstringspaces=false,
	breakatwhitespace=true,
	stringstyle = {\color{CtpGreen}},
	commentstyle={\color{CtpOverlay1}},
	basicstyle = {\footnotesize\color{CtpText}\ttfamily},
	keywordstyle = {\color{CtpMauve}},
	keywordstyle = [2]{\color{CtpBlue}},
	keywordstyle = [3]{\color{CtpYellow}},
	keywordstyle = [4]{\color{CtpLavender}},
	keywordstyle = [5]{\color{CtpPeach}},
	keywordstyle = [6]{\color{CtpTeal}},
	otherkeywords = {<, ||, =, ?},
	morekeywords = [2]{new, create, present, email, description, creator, protect_from_forgery, before_action},
	morekeywords = [3]{PageController, ApplicationController, Page},
	morekeywords = [4]{@page},
	morekeywords = [5]{exception, do_some_for_pages, @page, @admin},
	morekeywords = [6]{<, ||, =, ?},
}

% Set the title and author, utilizing CtpPink
\title{ \Huge \textbf{\textcolor{CtpPink}{Catppuccin Theme for} \\ \textcolor{CtpLavender}{\LaTeX{} Documents}} \vspace{-3em}}
\date{}

% Start our dock
\begin{document}

\maketitle

\section{\textcolor{CtpSky}{Simple Paragraph}}
\textcolor{CtpYellow}{This is what a simple paragraph looks like.} You can see that all the colors in this pallet complement each other and maintain an elegance all within the same proximity. The default text color is called \textbf{\textcolor{CtpGreen}{CtpText}} while the background color is called \textbf{\textcolor{CtpGreen}{CtpBase}}.

\begin{minipage}{0.47\linewidth}
	\section{\textcolor{CtpSky}{Hyperref}}
	(internal) Link: \pageref{l}

	Citation: \cite{bb}

	File: \href{file:///home/second_user}{goto file}

	Menu: Only in Acrobat menu items

	Run: no idea where this color is used

	Url: \url{https://github.com}

	\label{l}
	\begin{thebibliography}{2}
		\bibitem{bb} example bib item.
	\end{thebibliography}
\end{minipage}
\hfill\vline\hfill
\begin{minipage}{0.47\linewidth}
	\section{\textcolor{CtpSky}{Math}}

	You even can make math formulas, equations, and examples look nicer and more legible:


	\[\large{
		\textcolor{CtpBlue}{\lim_{{n \to \infty}} \textcolor{CtpPeach}{\int_{\textcolor{CtpRed}{a}}^{\textcolor{CtpRed}{b}} \frac{\textcolor{CtpGreen}{\sin}(\textcolor{CtpYellow}{nx})}{\textcolor{CtpYellow}{x}} \, \textcolor{CtpTeal}{dx}}}
	}\]

	\tiny{(Or just color dump like I did)}
\end{minipage}
\\

\section{\textcolor{CtpSky}{Code Snippets}}

\begin{lstlisting}[language=Ruby,style=ruby_on_rails, caption={A ruby on rails code sample}]
class PageController < ApplicationController
  protect_from_forgery with: :exception
  before_action :do_some_for_pages

  '''
  This action creates a new page
  '''
  def new
    @page = Page.new
    creator = current_user || @admin

    # Check if a creator is present
    if creator.present?
      @page.creator = creator.email
    end

    @page.description = 'This is the default description'
  end
end
\end{lstlisting}

\end{document}


